\documentclass{article}
\author{The Contributors}
\title{My Glimpse Of Propositional Logic Inside The Ivory Tower Of
  Mathematics}
  \usepackage[pdftex]{hyperref}
\begin{document}
  \maketitle
  \tableofcontents

  \section{Introduction}

  I'm a mathematical barbarian, but a few days ago I broke into the ivory tower
  of mathematics and learnt the basics of Propositional Logic. In order to help
  me learn it, I wrote this tutorial. It helped me, and perhaps you'll find it
  useful too.

  \subsection{Licence}

    This work is licensed under a
    \href{http://creativecommons.org/licenses/by-sa/4.0/}{Creative Commons
    Attribution-ShareAlike 4.0 International License}.

  \subsection{Contributing}

    All contributions (corrections, amendments, additions etc) welcome at
    \url{https://github.com/tlocke/maths}.


  \section{\label{sec:propositions}Propositions}

    In propositional logic, we're only interested in sentences that are either
    true or false. Such sentences are known as \emph{propositions}. Here are
    some sentences together with explanations of whether they are propositions
    or not:

    \begin{itemize}

      \item \emph{How many miles to Crinnis?}

        This is a question, so not a proposition.

      \item \emph{Elephants have four legs.}

        Yay, an actual proposition! It's a sentence that is either true or
        false.

      \item \emph{Don't dilly-dally.}

        This isn't true or false, it sounds like someone admonishing somebody.
        Not a proposition.

      \item \emph{The capital of France is Paris.}

        Yes, a proposition.

      \item \emph{He likes chocolate.}

        This sounds like a proposition, but according to those logicians in
        the ivory tower it doesn't count because it relies on knowing who `he'
        is.

      \item \emph{Don't spoil the ship for a ha'peth of tar.}

        This is a proverb, not a proposition.

    \end{itemize}

    I've got this nagging doubt in my mind. Most propositions I can think of
    aren't \emph{totally} ambiguous. Take the `Elephants have four legs'
    example. Maybe there's a three legged elephant in existence, perhaps one in
    a zoo got gangrene or something and had to have a leg amputated \ldots
    Nevertheless, let's suspend our disbelief and imagine all those perfect
    propositions.

    \subsection*{Questions}

      \begin{enumerate} 

        \item Which of the following are propositions?
          \begin{enumerate}
            \item Who is John Galt?
            \item He's over there.
            \item Three divided by three is one.
            \item Belgium is a European country.
            \item Praise be!
            \item Blue is a colour.
          \end{enumerate}

        \item Are the following propositions true or false?
          \begin{enumerate}
            \item Four is greater than two.
            \item Tennis is a colour.
            \item A square has eight sides.
            \item A cube has eight corners.
            \item Birmingham is a city in England.
            \item The word `rotavator' is a palindrome.
          \end{enumerate}

      \end{enumerate}

  \appendix
  \section{Answers To Section Questions}

  \subsection*{\ref{sec:propositions} Propositions}

      \begin{enumerate} 
        \item
          \begin{enumerate}
            \item Not a proposition.
            \item Not a proposition.
            \item A proposition.
            \item A proposition.
            \item Not a proposition.
            \item A proposition.
          \end{enumerate}
        \item
          \begin{enumerate}
            \item True.
            \item False.
            \item False.
            \item True.
            \item True.
            \item True.
          \end{enumerate}
      \end{enumerate}
      
\end{document}
